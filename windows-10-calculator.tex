%% windows-10-calculator.tex
%% Copyright 2023 DontEatOreo
%
% This work may be distributed and/or modified under the
% conditions of the LaTeX Project Public License, either version 1.3
% of this license or (at your option) any later version.
% The latest version of this license is in
%   https://www.latex-project.org/lppl.txt
% and version 1.3c or later is part of all distributions of LaTeX
% version 2008 or later.
%
% This work has the LPPL maintenance status `maintained'.
% 
% The Current Maintainer of this work is DontEatOreo.
%
% This work consists of the files style.sty

\documentclass[12pt]{article}

%% style.tex
%% Copyright 2023 DontEatOreo
%
% This work may be distributed and/or modified under the
% conditions of the LaTeX Project Public License, either version 1.3
% of this license or (at your option) any later version.
% The latest version of this license is in
%   https://www.latex-project.org/lppl.txt
% and version 1.3c or later is part of all distributions of LaTeX
% version 2008 or later.
%
% This work has the LPPL maintenance status `maintained'.
% 
% The Current Maintainer of this work is DontEatOreo.

% Packages that magically make LaTeX better
\usepackage{microtype}

% For customizing Headers and Footers
\usepackage{fancyhdr}

% to create footnotes
\usepackage{footnote}

% ----------MARGINS----------

\usepackage[
    % letterpaper, % For North American readers
    a4paper, % For European readers
    % left=3.5cm, % For spiral or comb binding
    left=2.5cm, % For saddle stitch or staple binding
    right=3cm,
    top=3cm,
    bottom=3cm
]{geometry}

% ----------END MARGINS----------

% ----------FONTS----------

% Define the font family as Computer Modern
\usepackage{lmodern}

% Define the font sizes for different text elements
\usepackage{anyfontsize}
\usepackage{sectsty}
\sectionfont{\fontsize{14}{18}\selectfont}
\subsectionfont{\fontsize{12}{16}\selectfont}
\subsubsectionfont{\fontsize{10}{14}\selectfont}

% Define the font styles for different text elements
\usepackage{titlesec}
\titleformat{\section}{\normalfont\bfseries}{\thesection}{1em}{}
\titleformat{\subsection}{\normalfont\bfseries}{\thesubsection}{1em}{}
\titleformat{\subsubsection}{\normalfont\bfseries}{\thesubsubsection}{1em}{}

% ----------END FONTS----------


% ----------HEADINGS----------

% Define the font, size, color, and capitalization rules for each level of heading
\titleformat{\part}[display]{\normalfont\huge\bfseries\centering}{\thepart}{20pt}{\Huge}
\titleformat{\chapter}[display]{\normalfont\Large\bfseries}{\thechapter}{20pt}{\LARGE}
\titleformat{\section}{\normalfont\large\bfseries}{\thesection}{1em}{}
\titleformat{\subsection}{\normalfont\normalsize\bfseries}{\thesubsection}{1em}{}
\titleformat{\subsubsection}{\normalfont\normalsize\bfseries}{\thesubsubsection}{1em}{}
% Add more levels of headings as needed

% Define the indentation for each level of heading
\titlespacing*{\part}{0pt}{50pt}{40pt}
\titlespacing*{\chapter}{0pt}{50pt}{40pt}
\titlespacing*{\section}{0pt}{3.5ex plus 1ex minus .2ex}{2.3ex plus .2ex}
\titlespacing*{\subsection}{15pt}{3.25ex plus 1ex minus .2ex}{1.5ex plus .2ex}
\titlespacing*{\subsubsection}{30pt}{3.25ex plus 1ex minus .2ex}{1.5ex plus .2ex}
% Add more levels of headings as needed

% Set the depth of numbering for the headings
% The default value is 2, which means up to subsections are numbered
% You can change this value to a higher or lower number as needed
\setcounter{secnumdepth}{2}

% ----------END HEADINGS----------

% ----------TABLES----------

% For creating tables
\usepackage{tabularx}
\usepackage{booktabs}

% For adjusting the font size of the table
\usepackage{adjustbox}

% For rotating the table
\usepackage{rotating}

% For adding captions to the table
\usepackage{caption}

% Use the € symbol for prices
\usepackage{eurosym}

% Define a global style for the table
\newcolumntype{L}{>{\raggedright}X} % This line is commented out because it is not used in your code.
\setlength{\tabcolsep}{5pt} % This line is commented out because it is not used in your code.
\renewcommand{\arraystretch}{1.2} % This line is commented out because it is not used in your code.
\captionsetup[table]{font=small,skip=0pt} % This line is commented out because it is not used in your code.

% ----------END OF TABLES----------

% ----------LISTS----------

% Use the enumitem package to customize the bullet and numbered lists
\usepackage{enumitem}

\setlist[enumerate,1]{label=\arabic*.,ref=\arabic*}
\setlist[enumerate,2]{label=\theenumi\alph*.,ref=\theenumi\alph*}
\setlist[enumerate,3]{label=\theenumii(\roman*),ref=\theenumii(\roman*)}

% ----------END LISTS----------

% Set Footnotes at the bottom
\usepackage[bottom]{footmisc}

% Add proprer spacing
\usepackage{parskip}

% For Hyperlinks and Bookmarks
% You want to load this last
\usepackage{hyperref}


% ----------REFERENCES----------

\usepackage{biblatex}
\bibliography{bib/windows-10-calculator.bib}

% ----------END OF REFERENCES----------

\title{Windows 10 Calculator Guide}
\author{DontEatOreo}
\date{10-21-2023}

\begin{document}

\maketitle

\pagebreak

\tableofcontents

\pagebreak

\section{Introduction}
The Windows 10 Calculator is a free and easy-to-use app that allows you to perform various calculations on your PC or tablet.\footnote{You can download the app for free from the Microsoft Store.} You can use it for simple arithmetic, scientific calculations, unit conversions, date calculations, and more. You can also switch between different modes and themes to suit your needs and preferences

This guide will help you learn how to use the Windows 10 Calculator app and its features. It is intended for casual users who want to perform basic or common calculations on their devices. You do not need any prior knowledge or experience with the app to follow this guide.

\section{Getting Started}
To launch the Windows 10 Calculator app, you can do one of the following:
\begin{itemize}
    \item Click the Start button and type "Calculator" in the search box. Then, click the Calculator app icon from the list of results.
    \item Press the Windows logo key + R to open the Run dialog box. Then, type "calc" and press Enter or click OK.
    \item Pin the Calculator app to your taskbar or Start menu for quick access. To do this, right-click the Calculator app icon and select "Pin to taskbar" or "Pin to Start".
\end{itemize}

Once you open the app, you will see a simple calculator interface with a number pad, basic operators, and a display area. You can use your mouse, keyboard, or touch screen to enter numbers and perform calculations.

\section{Using Different Themes}
The Windows 10 Calculator app has two themes that you can use to change the appearance of the app. You can switch between themes by clicking the settings icon (gear) at the top right corner of the app window. You will see a list of themes that you can choose from:

\begin{itemize}
    \item Light: This is the default theme that uses a white background and black text for the app.
    \item Dark: This theme uses a black background and white text for the app.
\end{itemize}

To use a theme, simply select it from the settings menu and the app will apply it immediately. You can also change the theme based on your system settings by toggling the "Match system theme" option in the settings menu.

\section{Using Different Modes}
The Windows 10 Calculator app has several modes that you can use for different types of calculations. You can switch between modes by clicking the menu icon (three horizontal lines) at the top left corner of the app window. You will see a list of modes that you can choose from. In this section, we will explain how to use each mode and what features they offer.

\subsection{Using Standard Mode}
Standard mode is the default and simplest mode of the Windows 10 Calculator app. It is useful for basic math operations like adding, subtracting, multiplying, and dividing. You can also use Standard mode to keep the calculator window on top of other windows.

To use Standard mode, select the Start button, and then select Calculator in the list of apps. If you are not already in Standard mode, select the Open Navigation button to switch modes, and then select Standard\cite{calculator-app}. You can use your mouse, keyboard, or touch screen to enter numbers and operators in the calculator. You can also use some keyboard shortcuts to perform basic operations faster.

To keep the calculator window on top of other windows, click the keep-on-top icon next to the Standard title. It looks like an arrow pointing at a small box\cite{calculator-top}. This will pin the calculator to your screen and it will stay on top permanently. You can drag it around or resize it as needed. To unpin the calculator, click the same icon again.

You can use the following functions in standard mode\cite{calculator-how-to-use}:

\begin{itemize}
    \item \textbf{CE}: This button clears the current entry, meaning it deletes the last number or operator you entered.
    \item \textbf{C}: This button clears all, meaning it resets the calculator and deletes everything you entered.
    \item \textbf{Backspace}: This button deletes the last digit you entered.
    \item \textbf{+/-}: This button changes the sign of the current number from positive to negative or vice versa.
    \item \textbf{1/x}: This button calculates the reciprocal of the current number, meaning it divides one by that number.
    \item \textbf{x$^2$}: This button calculates the square of the current number, meaning it multiplies that number by itself.
    \item \textbf{$^2\sqrt{x}$}: This button calculates the square root of the current number, meaning it finds a number that when multiplied by itself gives that number.
    \item \textbf{\%}: This button calculates the percentage of the previous number using the current number as the rate. For example, if you enter 200 + 50, it will calculate 200 + (200 * 0.5) = 300.
    \item \textbf{MS}: This button stores the current number in memory for later use.
    \item \textbf{MR}: This button recalls the number stored in memory and displays it on the screen.
    \item \textbf{MC}: This button clears the memory and deletes the stored number.
    \item \textbf{M+}: This button adds the current number to the memory and stores the result in memory.
    \item \textbf{M-}: This button subtracts the current number from the memory and stores the result in memory.
\end{itemize}

\pagebreak

\subsection{Scientific Mode}
Scientific mode allows you to perform advanced mathematical operations such as trigonometry, logarithms, exponentiation, factorials, and more. You can also use constants, angles, and base conversions in this mode.

To use scientific mode, select it from the menu icon at the top left corner of the app window. The app will display additional buttons for scientific functions on the left side of the number pad. You can enter the numbers and functions that you want to calculate using your mouse, keyboard, or touch screen. The app will display the result in the display area. You can also use the backspace button to delete the last input or the clear button to clear the entire input.

You can use the following functions in scientific mode\cite{calculator-how-to-use}:

\begin{itemize}
    \item Trigonometry: You can use the sine (sin), cosine (cos), tangent (tan), cosecant (csc), secant (sec), and cotangent (cot) buttons to calculate the trigonometric ratios of an angle. You can also use their inverse functions by clicking the inv button before clicking them. For example, $sin(30) = 0.5$ and inv $sin(0.5) = 30$. You can also use the hyperbolic (HYP) button to switch between regular and hyperbolic trigonometric functions. For example, $sinh(1) = 1.175201194$ and $cosh(1) = 1.543080635$.
    \item Logarithms: You can use the logarithm (log) and natural logarithm (ln) buttons to calculate the logarithms of a number with base 10 and base e respectively. You can also use their inverse functions by clicking the inv button before clicking them. For example, $log(100) = 2$ and inv $log(2) = 100$.
    \item Exponentiation: You can use the exponentiation ($x^y$) and power of e ($e^x$) buttons to calculate a number raised to a power or a power of e respectively. You can also use their inverse functions by clicking the inv button before clicking them. For example, $2^3$ = 8 and inv $x^y(8) = 3$.
    \item Factorials: You can use the factorial (!) button to calculate the factorial of a positive integer or a decimal number using the gamma function. For example, $5! = 120$ and $2.5! = 3.32335097$.
    \item Constants: You can use the pi ($\pi$) and e buttons to insert the values of pi and e respectively into your calculation. For example, $\pi * r * r = area of a circle$.
    \item Angles: You can use the degrees (DEG), radians (RAD), and gradians (GRAD) buttons to switch between different units of angle measurement for trigonometric functions. For example, $sin(30) = 0.5$ in degrees and $sin(\pi/6) = 0.5$ in radians.
    \item Base conversions: You can use the hexadecimal (HEX), decimal (DEC), octal (OCT), and binary (BIN) buttons to switch between different number systems for your calculation. For example, $10$ in decimal is A in hexadecimal and $1010$ in binary.
    \item Modulo: You can use the modulo (Mod) button to calculate the remainder of a division. For example, $7$ Mod $3 = 1$.
    \item Root: You can use the root ($\sqrt[x]{y}$) button to calculate any root of a number. For example, $\sqrt[3]{8} = 2$.
    \item Parentheses: You can use the parentheses () buttons to group expressions and change the order of operations. For example, $(2+3)*4 = 20$ and $2+(3*4) = 14$.
\end{itemize}

\pagebreak

\subsection{Programmer Mode}
Programmer mode allows you to perform calculations using binary, octal, decimal, and hexadecimal number systems. You can also use bitwise operations, logical operations, shifts, and rotations in this mode.

To use programmer mode, select it from the menu icon at the top left corner of the app window. The app will display additional buttons for programmer functions on the left side of the number pad. You can enter the numbers and functions that you want to calculate using your mouse, keyboard, or touch screen. The app will display the result in the display area. You can also use the backspace button to delete the last input or the clear button to clear the entire input.

You can use the following functions in programmer mode\cite{calculator-how-to-use}:

\begin{itemize}
    \item Bitwise operations: You can use the bitwise AND ($\&$), OR ($|$), XOR ($\hat{}$), and NOT ($\sim$) buttons to perform bitwise operations on two numbers. For example, $1010 \& 1100 = 1000$ and $\sim1010 = 0101$.
    \item Logical operations: You can use the logical AND ($\&\&$), OR ($||$), XOR ($\neq$), and NOT (!$\ $) buttons to perform logical operations on two numbers. For example, $1 \&\& 0 = 0$ and $!\ 1 = 0$.
    \item Shifts: You can use the left shift ($<<$) and right shift ($>>$) buttons to shift a number by a specified number of bits to the left or right respectively. For example, $1010$ $<<$ $2 = 101000$ and $1010$ $>>$ $2$ = $10$.
    \item Rotations: You can use the rotate left (ROL) and rotate right (ROR) buttons to rotate a number by a specified number of bits to the left or right respectively. For example, $ROL(1010,2) = 101000$ and $ROR(1010,2) = 10000010$.
    \item Base conversions: You can use the hexadecimal (HEX), decimal (DEC), octal (OCT), and binary (BIN) buttons to switch between different number systems for your calculation. For example, $10$ in decimal is A in hexadecimal and $1010$ in binary.
    \item Word sizes: You can use the word (WORD), double word (DWORD), and quad word (QWORD) buttons to switch between different word sizes for your calculation. For example, $1010$ in QWORD is $$000000000000000000000000000000000000000000000000000000001010$$.
    \item Hexadecimal digits: You can use the A-F buttons to enter hexadecimal digits in your calculation. For example, $A + B = 15$ in hexadecimal.
    \item Parentheses: You can use the parentheses () buttons to group expressions and change the order of operations. For example, $(1010 \& 1100) | 1111 = 1111$.
\end{itemize}

\pagebreak

\subsection{Date Calculation Mode}
Date calculation mode allows you to calculate the difference between two dates or add or subtract days from a given date. You can also use calendar functions such as leap year check and week number in this mode.

To use date calculation mode, select it from the menu icon at the top left corner of the app window. The app will display two calendars on the left side of the app window and a result area on the right side. You can enter the dates that you want to calculate using your mouse, keyboard, or touch screen. The app will display the result in the result area. You can also use the backspace button to delete the last input or the clear button to clear the entire input.

You can use the following functions in date calculation mode\cite{calculator-how-to-use}:

\begin{itemize}
    \item Difference between dates: You can calculate the difference between two dates by selecting them from the calendars or typing them in the date fields. The app will display the difference in years, months, weeks, and days in the result area. For example, if you select January 1, 2020 and January 1, 2021, the app will display 1 year, or 12 months, or 52 weeks and 2 days.
    \item Add or subtract days: You can add or subtract days from a given date by selecting it from the calendar or typing it in the date field. Then, you can enter the number of days that you want to add or subtract using your mouse, keyboard, or touch screen. The app will display the resulting date in the result area. For example, if you select January 1, 2020 and enter +365, the app will display January 1, 2021.
    \item Leap year check: You can check if a given year is a leap year by selecting it from the calendar or typing it in the year field. The app will display "Yes" or "No" in the result area. For example, if you select or type 2020, the app will display "Yes".
    \item Week number: You can find out which week of the year a given date belongs to by selecting it from the calendar or typing it in the date field. The app will display the week number in ISO format (YYYY-Www-D)    want to convert from and to. For example, Swap currencies(USD to EUR) will swap US dollars to euros and vice versa.
    \item Historical rates: You can use the historical rates button (Historical rates) to view the exchange rates of a currency pair for the past 30 days. You can also select a specific date from the calendar to see the rate for that day. For example, Historical rates(USD to EUR) will show the exchange rates of US dollars to euros for the past 30 days, and you can select 01/01/2023 to see the rate for that day.
    \item Inverse conversions: You can use the inverse conversions button (Inverse) to perform inverse conversions between currencies. For example, Inverse(USD to EUR) will convert euros to US dollars instead of US dollars to euros.
\end{itemize}


\pagebreak

\section{Using the Calculator History}

The calculator history lets you view, copy, or clear your previous calculations. You can use the history to recall numbers or catch errors.

To access the history, resize the calculator app window horizontally far enough that you see the History section on the right side. If it's too narrow, you can click the History icon at the top-right to show a slide-out panel, but this isn't as convenient.\footnote{If it's too narrow, you can click the History icon at the top-right to show a slide-out panel, but this isn't as convenient.}

In the history section, you can see a list of your past calculations, along with their results. You can click on any calculation to copy it to the clipboard, or click on the result to paste it into the current calculation. You can also use the keyboard shortcuts Ctrl+H to open or close the history panel, and Ctrl+Shift+D to clear the history.

\pagebreak

\section{Using the Memory Functions}
The memory functions let you store and recall values in memory. You can use the memory functions to perform calculations with multiple numbers without losing track of them.

To use the memory functions, you need to switch to standard mode first. Then, you will see five memory buttons (MC, MR, M+, M-, MS) above the number pad. You can use these buttons to perform the following actions:

\begin{itemize}
    \item MC: Clears the memory.
    \item MR: Recalls the memory value.
    \item M+: Adds the current value to the memory value.
    \item M-: Subtracts the current value from the memory value.
    \item MS: Stores the current value in memory.
\end{itemize}

For example, if you want to calculate $(2+3)*4+(5-6)*7$, you can use the memory functions as follows:

\begin{enumerate}
    \item Enter $2+3$ and press MS. This will store 5 in memory.
    \item Enter 4 and press M+. This will add 4 to the memory value and store 9 in memory.
    \item Enter 5-6 and press MS. This will store -1 in memory and overwrite the previous value.
    \item Enter 7 and press M+. This will add 7 to the memory value and store 6 in memory.
    \item Press MR. This will recall 6 from memory and display it as the result.
\end{enumerate}

\pagebreak

\section{Using Keyboard Shortcuts}
The Windows 10 Calculator app supports various keyboard shortcuts that can help you perform calculations faster and easier. You can use the keyboard shortcuts to enter numbers, operators, functions, modes, themes, and more. In this section, we will list some of the common keyboard shortcuts that you can use in the app.

\subsection{Using Memory and History Functions}
You can use the following keyboard shortcuts to use the memory and history functions in the app\cite{calculator-shortcuts}:

\begin{itemize}
    \item \textbf{Ctrl+M}: Add result to memory pannel.
    \item \textbf{Ctrl+L}: Clear the memory.
    \item \textbf{Ctrl+H}: Open or close the history panel.
    \item \textbf{Ctrl+Shift+D}: Clear the history.
\end{itemize}

\subsection{Switching Between Modes}
You can use the following keyboard shortcuts to switch between different modes in the app\cite{calculator-shortcuts}:

\begin{itemize}
    \item \textbf{Alt+1}: Switch to Standard mode.
    \item \textbf{Alt+2}: Switch to Scientific mode.
    \item \textbf{Alt+3}: Switch to Graphing mode.
    \item \textbf{Alt+4}: Switch to Programmer mode.
    \item \textbf{Alt+5}: Switch to Date Calculation mode.
\end{itemize}

\subsection{Using Basic Operations}
You can use the following keyboard shortcuts to perform basic operations in the app\cite{calculator-shortcuts}:

\begin{itemize}
    \item \textbf{Enter}: Select = in Standard mode, Scientific mode, and Programmer mode.
    \item \textbf{Delete}: Clear current input (select CE).
    \item \textbf{Esc}: Fully clear input (select C).
    \item \textbf{F9}: Select $+/-$ in Standard mode, Scientific mode, and Programmer mode.
    \item \textbf{R}: Select $1/x$ in Standard mode and Scientific mode.
    \item \textbf{@}: Select $2\sqrt{x}$ in Standard mode and Scientific mode.
    \item \textbf{\%}: Select $\%$ in Standard mode and Programmer mode.
    \item \textbf{R}: Select 1/x in Standard mode and Scientific mode.
\end{itemize}

\pagebreak

\subsection{Using Programmer Functions}
You can use the following keyboard shortcuts to use programmer functions in the app\cite{calculator-shortcuts}:

\begin{itemize}
    \item \textbf{F2}: Select DWORD in Programmer mode
    \item \textbf{F3}: Select WORD in Programmer mode
    \item \textbf{F4}: Select BYTE in Programmer mode
    \item \textbf{F5}: Select HEX in Programmer mode
    \item \textbf{F6}: Select DEC in Programmer mode
    \item \textbf{F7}: Select OCT in Programmer mode
    \item \textbf{F8}: Select BIN in Programmer mode
    \item \textbf{F12}: Select QWORD in Programmer mode
\end{itemize}

\subsection{Using Scientific Functions}
You can use the following keyboard shortcuts to use scientific functions in the app\cite{calculator-shortcuts}:

\begin{itemize}
    \item \textbf{G}: Select $2x$ in Scientific mode.
    \item \textbf{S}: Select $10^x$ in Scientific mode.
    \item \textbf{T}: Select $\tan$ in Scientific mode.
    \item \textbf{O}: Select $\cos$ in Scientific mode.
    \item \textbf{U}: Select $\sec$ in Scientific mode.
    \item \textbf{I}: Select $\csc$ in Scientific mode.
    \item \textbf{J}: Select $\cot$ in Scientific mode.
    \item \textbf{M}: Select $dms$ in Scientific mode.
    \item \textbf{N}: Select $\ln$ in Scientific mode.
    \item \textbf{P}: Select $\pi$ in Scientific mode.
    \item \textbf{Q}: Select $x^2$ in Standard mode and Scientific mode.
    \item \textbf{V}: Toggles on/off F-E buttonin Scientific mode
    \item \textbf{X}: Select $\exp$ in Scientific mode.
    \item \textbf{Y}: Select $x^y$ in Scientific mode.
    \item \textbf{\#}: Select $x^3$ in Scientific mode.
    \item \textbf{!}: Select $n!$ in Scientific mode.
    \item \textbf{\%}: Select $mod$ in Scientific mode.
\end{itemize}

\pagebreak

\printbibliography

\end{document}
